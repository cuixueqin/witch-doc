
%\renewcommand*{\@fnsymbol}[1]{\ifcase#1\or*\else\@arabic{\numexpr#1-1\relax}\fi}

\title{The WITCH 2016 Model - Documentation and Implementation of the Shared
Socioeconomic Pathways}


\author{Johannes Emmerling\thanks{Fondazione Eni Enrico Mattei (FEEM) and Centro Euromediterraneo sui
Cambiamenti Climatici (CMCC), Corso Magenta 63, 20123 Milano, Italy.}~\thanks{E-Mail: johannes.emmerling@feem.it. For any inquiries about the WITCH
model, please contact witch@feem.it.}, Laurent Drouet$^{*}$, Lara Aleluia Reis$^{*}$, Michela Bevione$^{*}$,
\and Loic Berger$^{*}$\thanks{Bureau F�d�ral du Plan, Avenue des Arts, 47-49, 1000 Bruxelles, Belgium.}
, Valentina Bosetti$^{*}$\thanks{Bocconi University, Via Roberto Sarfatti, 25, 20136 Milano, Italy.}
, Samuel Carrara$^{*}$, Enrica De Cian$^{*}$, \and Gauthier De
Maere D'Aertrycke$^{*}$\thanks{Engie, Boulevard Simon Bolivarlaan 34, 1000 Bruxelles, Belgium.}
, Tom Longden$^{*}$\thanks{University of Technology Sydney, 15 Broadway, Ultimo NSW 2007, Australia.}
, Maurizio Malpede$^{*}$, \and Giacomo Marangoni$^{*}$, Fabio Sferra$^{*}$\thanks{Climate Analytics, Friedrichstra�e 231, 10969 Berlin, Germany.}
~, Massimo Tavoni$^{*}$\thanks{Politecnico di Milano, Via Raffaele Lambruschini, 4/B, 20156 Milano,
Italy.} ~, \and Jan Witajewski-Baltvilks$^{*}$\thanks{Institute for Structural Research, Rejtana 15 lok. 28, 02-516 Warsaw,
Poland. }, Petr Havlik\thanks{International Institute for Applied Systems Analysis (IIASA), Schlossplatz
1, 2361 Laxenburg, Austria.}}
\maketitle
\begin{abstract}
This paper describes the WITCH - World Induced Technical Change Hybrid
- model in its structure, calibration, and the implementation of the
SSP/RCP scenario implementation. The WITCH model is a regionally disaggregated
hard-linked model based on a a Ramsey type optimal growth model and
a detailed bottom-up energy sector model. A particular focus of the
model is the modeling or technical change and RnD investments and
the analysis of cooperative and non-cooperative climate policies.
Moreover, the WITCH 2016 version now includes land-use change modeling
based on the GLOBIOM model, and air pollutants, as well as detailed
modeling of the transport sector and the possibility for stochastic
modeling. This version has been also used to implement the Shared
Socioeconomic Pathways (SSPs) set of scenarios and RCP based climate
policies to provide a new set of climate scenarios. In this paper,
we describe in detail the mathematical formulation of the WITCH model,
the solution method and calibration, as well as the implementation
of the five SSP scenarios. This report therefore provides detailed
information for interested users of the model, and for understanding
the implementation of the different ``worlds'' of the SSP.

\textbf{Keywords:} Integrated Assessment Model, SSPs, Climate Change,
Scenarios

\textbf{JEL Classification: }Q54, C63

quote as

Emmerling J, Drouet L, Aleluia Reis L, Bevione M, Berger L, Bosetti
V, Carrara S, De Cian E, De Maere D'Aertrycke G, Longden T, Malpede
M, Marangoni G, Sferra F, Tavoni M, Witajewski-Baltvilks J, Havlik
P (2016) The WITCH 2016 Model - Documentation and Implementation of
the Shared Socioeconomic Pathways, FEEM Nota di Lavoro 42.2016.

or

Emmerling J, Drouet L, et al. (2016) The WITCH 2016 Model - Documentation
and Implementation of the Shared Socioeconomic Pathways, FEEM Nota
di Lavoro 42.2016.\end{abstract}

